\chapter{Research methodology}
\section{Research}
\label{sec-Research}
\subsection{Research questions}
The upcoming research will investigate the following questions:  

\begin{mdframed}[backgroundcolor=lightgray!20]
\par RQ1: Do classes learned from a continual learning approach align with existing gravity spy classes?\\
RQ2: How effective are continual learning approaches when incorporating auxiliary channel data for detecting and classifying glitch morphologies?\\
RQ3: What are the main challenges in implementing continual learning strategies and how are they mitigated? 
\end{mdframed}

\subsection{Research method}
\label{subsec-researchmethod}
\subsubsection{RQ1}
In order to evaluate the alignment between classes acquired from a \acrlong{cl} approach and existing gravity spy classes, we proceed with the following steps:
\begin{itemize}
    \item Data preprocessing is required for the utilization of a \acrshort{cl} approach, involving the selection of either the O3a or O3b dataset that includes labeled data with various glitch categories. This dataset will be segmented into sequential chunks to represent a continuous data stream.
    \item Model selection and Training: A proper \acrshort{cl} algorithm will be chosen, taking into account aspects such as memory usage and its ability to reduce catastrophic forgetting. The model will be trained gradually on the segmented data, mimicking a continual learning scenario.
    \item In order to assess the alignment, we will compare the classes acquired from the \acrshort{cl} model with the existing gravity spy classes. This assessment could include traditional measures like accuracy and confusion matrix, as well as the specific metrics for \acrshort{cl} outlined in Section \ref{sec:il_metrics}. 
\end{itemize}
\subsubsection{RQ2}
In order to assess how well auxiliary channel-based strategies work for detecting and categorizing glitch shapes, we will utilize the following approach:
\begin{itemize}
    \item The dataset preparation will be done by gathering glitch-affected and clean signal samples. Useful information needs to be extracted from a selection of auxiliary channels in a limited time window around glitch transients. The approach of \citep{colgan2020efficient} will be used. 
    \item A multimodal fusion achitecture will be used to incorporate auxiliary channel data alongside the primary spectrogram dataset (main signal information). The model is trained on the selected dataset to learn robust glitch detection and classification. 
    \item Evaluating the model's performance is done by using metrics like precision, recall and F1-score. Additionally visualization techniques can be employed to understand the relative feature importance of selected auxiliary channels for glitch classification. 
\end{itemize}
\subsubsection{RQ3}
To investigate the main challenges in implementing \acrfull{cl} and explore mitigation strategies, we will take a two-pronged approach: 
\begin{itemize}
    \item We conduct a comprehenisve review of existing literature on \acrshort{cl} algorithms and their limitations. This helps to identify recurring challenges, such as catastrophic forgetting and resource constraints. 
    \item We might conduct controlled experiments with different \acrshort{cl} algorithms on computer vision benchmark datasets like Split MNIST/CIFAR, CORe50, ImageNet-Incremental \citep{wu2019large} or NLP datasets like WikiText-103 \citep{wang2021wikigraphs}, SentiWordNet \citep{das2010sentiwordnet}.
\end{itemize}