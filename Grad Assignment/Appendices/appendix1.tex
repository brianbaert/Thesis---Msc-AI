\chapter{MultiViewColorNet\_resnet18}
\label{appendix1}
\begin{verbatim}
class MultiViewColorNet_resnet18(nn.Module):
  def __init__(self, num_classes=10):
    # Call the parent class's constructor
    super(MultiViewColorNet_resnet18, self).__init__()
    # Initialize a pretrained ResNet-18 model with adjusted input size
    self.resnet = models.resnet18(weights='DEFAULT')
    # Access the first convolutional layer
    first_conv = self.resnet.conv1
    # Modify the kernel size of the first convolution
    first_conv.kernel_size = (7, 7)
    # Unfreeze all parameters in the model for training
    for param in self.resnet.parameters():
      param.requires_grad = True
    # Get the number of features in the last layer of the model
    num_features_in = self.resnet.fc.in_features
    # Replace the last layer with a new linear layer
    self.resnet.fc = nn.Linear(num_features_in, 120)
    # Add a second fully connected layer
    self.fc2 = nn.Linear(120, 84)
    # Add a third fully connected layer for the num_classes classes 
    in the GravitySpy dataset
    self.fc3 = nn.Linear(84, num_classes)
    # Add a dropout layer to prevent overfitting
    self.dropout = nn.Dropout(p=0.3)
    # Add a batch normalization layer
    self.bn = nn.BatchNorm1d(120)

  def forward(self, x):
    # Forward pass: compute the output of the model by passing 
    the input through the model
    x = self.resnet(x)
    # Apply batch normalization
    x = self.bn(x)
    # Apply the ReLU activation function
    x = F.relu(self.fc2(x))
    # Apply dropout
    x = self.dropout(x)
    # Pass the result through the final fully connected layer
    x = self.fc3(x)
    # Return the model's output
    return x
\end{verbatim}