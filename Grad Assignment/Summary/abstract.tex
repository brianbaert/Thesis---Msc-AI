\section*{Abstract}
This thesis explores the incorporation of \acrfull{cl} into traditional \acrfull{dl} frameworks for classifying glitches in \acrfull{gw} data analysis. The research utilizes strain data and auxiliary channel data in a multimodal fusion approach to enhance the performance of the glitch classification. Various \acrlong{cl} strategies, such as Naive CL, \acrfull{lwf}, \acrfull{agem}, \acrfull{ewc}, \acrfull{der}, \acrfull{der++}, and \acrfull{scr}, were assessed. The results reveal that although naive strategies exhibited the highest individual performance, multimodal fusion significantly boosts overall classification accuracy, reaching up to 100\% test set accuracy when combining spectrograms (Q-scans) and fractal dimension matrices. Furthermore, the thesis underscores the limitations and potential of the Avalanche framework for managing multimodal data, indicating the need for further research into CL applications in this area. Future research should aim to broaden the range of glitch types and explore advanced architectures like Transformers to further enhance the robustness and applicability of the proposed methods.
