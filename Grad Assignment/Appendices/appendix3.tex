\chapter{FractalDimensionConvNet}
\label{appendix3}
\begin{verbatim}
class FractalDimensionConvNet(nn.Module):
    def __init__(self):
        super(FractalDimensionConvNet, self).__init__()
        self.conv1 = nn.Conv1d(in_channels=50, out_channels=128, 
        kernel_size=3, stride=1, padding=1) #adjust from 347 to 50
        self.bn1 = nn.BatchNorm1d(128)
        self.conv2 = nn.Conv1d(in_channels=128, out_channels=64, 
        kernel_size=3, stride=1, padding=1)
        self.bn2 = nn.BatchNorm1d(64)
        self.conv3 = nn.Conv1d(in_channels=64, out_channels=32, 
        kernel_size=3, stride=1, padding=1)
        self.bn3 = nn.BatchNorm1d(32)
        self.conv4 = nn.ConvTranspose1d(in_channels=32, out_channels=64, 
        kernel_size=3, stride=1, padding=1)
        self.bn4 = nn.BatchNorm1d(64)
        self.fc1 = nn.Linear(64*56, 1024)
        #self.fc1 = nn.Linear(64*5, 1024) #adjust input size for fd statistics
        self.fc2 = nn.Linear(1024,256)
        self.dropout1 = nn.Dropout(0.1)
        self.dropout2 = nn.Dropout(0.3)
        self.fc3 = nn.Linear(256,3)

    def forward(self, x):
        x = nn.functional.selu(self.conv1(x))
        x = self.bn1(x)
        x = nn.functional.selu(self.conv2(x))
        x = self.bn2(x)
        x = nn.functional.selu(self.conv3(x))
        x = self.bn3(x)
        x = self.dropout1(x)
        x = nn.functional.selu(self.conv4(x))
        x = self.bn4(x)
        x = self.dropout2(x)
        x = x.view(x.size(0), -1)
        x = nn.functional.selu(self.fc1(x))
        x = nn.functional.relu(self.fc2(x))
        x = self.fc3(x)
        return x
\end{verbatim}